\documentclass{VUMIFPSkursinis}
\usepackage{algorithmicx}
\usepackage{algorithm}
\usepackage{algpseudocode}
\usepackage{amsfonts}
\usepackage{amsmath}
\usepackage{bm}
\usepackage{caption}
\usepackage{color}
\usepackage{float}
\usepackage{graphicx}
\usepackage{listings}
\usepackage{subfig}
\usepackage{wrapfig}
\usepackage{multirow}
\usepackage{array,makecell}

% Titulinio aprašas
\university{Vilniaus universitetas}
\faculty{Matematikos ir informatikos fakultetas}
\department{Programų sistemų katedra}
\papertype{Programų sistemų inžinerija: 1 laboratorinis darbas}
\title{Stalo žaidimų programėlė}
\titleineng{Board Games Application}
\status{2 kurso 5 grupės studentai}
\author{Elena Reivytytė}
\secondauthor{Matas Šilinskas}
\thirdauthor{Kasparas Taminskas}
\fourthauthor{Aidas Vaikšnoras}
\fifthauthor{Tadas Žaliauskas}
\supervisor{dr. Vytautas Valaitis}
\date{Vilnius – \the\year}

% Nustatymai
% \setmainfont{Palemonas}   % Pakeisti teksto šriftą į Palemonas (turi būti įdiegtas sistemoje)
\bibliography{bibliografija}

\begin{document}
\maketitle

\sectionnonum{Anotacija}
Šiame dokumente pristatomas mobiliosios stalo žaidimų programėlės Board Games 4+1 architektūrinis
modelis, kurį sudaro loginis, kūrimo, fizinis, procesų ir užduočių pjūviai, visi kartu parodantys
sistemą iš skirtingų pusių, papildantys vieni kitus. Modelio pagrindas - standartinės UML diagramos.

\tableofcontents

\sectionnonum{Įvadas}
Board Games - tai aplikacija, sujungianti norinčius žaisti stalo žaidimus žmones
su tais, kuriems trūksta žaidėjų. 


\section{Nefunkciniai reikalavimai}

\newcounter{nfrcount}
\newcommand\rownumber{\stepcounter{nfrcount}\arabic{nfrcount}}

\begin{tabular}{ | >{\centering}m{2cm} | m{10cm} | >{\centering}m{2.5cm} | } \hline
\multicolumn{3}{ |l| }{\textbf{\textit{Vidinių interfeisų reikalavimai}}} \tabularnewline \hline
\multicolumn{3}{ |l| }{\textbf{OS reikalavimai}} \tabularnewline \hline
NFR\rownumber & Aplikacija turi būti palaikoma IOS (nuo 8.0 versijos) ir Android (nuo 4.0 versijos) įrenginiuose & Būtina\tabularnewline \hline
NFR\rownumber & Aplikacija pageidautina, kad būtu palaikoma Windows Phone opracinėje sistemoje. & Pageidautinas\tabularnewline \hline
NFR\rownumber & Aplikacija turi būti pasiekiama ir be išmanaus telefono - per naršyklę & Būtina\tabularnewline \hline
NFR\rownumber & Išmanusis renginys turi turėti GPS modulį. & Būtina\tabularnewline \hline
NFR\rownumber & Išmanusis įrenginys turi turėti prieigą prie interneto & Būtina\tabularnewline \hline
\multicolumn{3}{ |l| }{\textbf{Saveikos su DB reikalavimai}} \tabularnewline \hline
NFR\rownumber & Duomenys saugomi naudojant MySQL (ne senesnė nei 5.0 versija) duomenų bazių valdymo sistemą & Būtina\tabularnewline \hline
NFR\rownumber & Turi egzistuoti lentelės: “Vartotojai”, “Žaidimai”, “Vartotojų sukurti žaidimai”, “Vartotojų žaidžiami žaidimai”. & Būtinas\tabularnewline \hline
NFR\rownumber & Turi būti neribojamas duomenų bazės duomenų pralaidumas. & Būtina\tabularnewline \hline
\multicolumn{3}{ |l| }{\textbf{Saveikos su DB reikalavimai}} \tabularnewline \hline
NFR\rownumber & Aplikacija su serveriu keičiasi duomenimis JSON formatu. & Būtina\tabularnewline \hline
\multicolumn{3}{ |l| }{\textbf{Darbo kompiuterių tinkluose reikalavima}} \tabularnewline \hline
NFR\rownumber & Duomenys siunčiami HTTP protokolu. & Būtina\tabularnewline \hline
\multicolumn{3}{ |l| }{\textbf{Saveikos su  kitomis programomis reikalavimai:}} \tabularnewline \hline
NFR\rownumber & Facebook ir Google api vartotojo autorizacijai & Būtina\tabularnewline \hline
NFR\rownumber & Aplikacija reikalauja GPS prieigos teisių. & Būtina\tabularnewline \hline
NFR\rownumber & Aplikacija reikalauja fotoaparato prieigos teisių. & Būtina\tabularnewline \hline
NFR\rownumber & Aplikacija reikalauja prieigos prei vartotojo įrenginyje turimų failų. & Būtina\tabularnewline \hline
NFR\rownumber & Aplikacija reikalauja priegos prie mobiliojo interneto, jei neprisijungta prie Wi-Fi. & Būtina\tabularnewline \hline
NFR\rownumber & Google Api žaidimo vietos nustatymui. & Būtina\tabularnewline \hline
\multicolumn{3}{ |l| }{\textbf{Programavimo aplinkos reikalavimai:}} \tabularnewline \hline
\end{tabular}

\end{document}
